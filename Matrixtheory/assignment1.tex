%iffalse
\let\negmedspace\undefined
\let\negthickspace\undefined
\documentclass[journal,12pt,twocolumn]{IEEEtran}
\usepackage{cite}
\usepackage{amsmath,amssymb,amsfonts,amsthm}
\usepackage{algorithmic}
\usepackage{multicol}
\usepackage{graphicx}
\usepackage{textcomp}
\usepackage{xcolor}
\usepackage{txfonts}
\usepackage{listings}
\usepackage{enumitem}
\usepackage{mathtools}
\usepackage{gensymb}
\usepackage{comment}
\usepackage[breaklinks=true]{hyperref}
\usepackage{tkz-euclide} 
\usepackage{listings}
\usepackage{gvv}                                        
%\def\inputGnumericTable{}                                 
\usepackage[latin1]{inputenc}                                
\usepackage{color}                                            
\usepackage{array}                                            
\usepackage{longtable}                                       
\usepackage{calc}                                             
\usepackage{multirow}                                         
\usepackage{hhline}                                           
\usepackage{ifthen}                                           
\usepackage{lscape}
\usepackage{tabularx}
\usepackage{array}
\usepackage{float}


\newtheorem{theorem}{Theorem}[section]
\newtheorem{problem}{Problem}
\newtheorem{proposition}{Proposition}[section]
\newtheorem{lemma}{Lemma}[section]
\newtheorem{corollary}[theorem]{Corollary}
\newtheorem{example}{Example}[section]
\newtheorem{definition}[problem]{Definition}
\newcommand{\BEQA}{\begin{eqnarray}}
\newcommand{\EEQA}{\end{eqnarray}}
\newcommand{\define}{\stackrel{\triangle}{=}}
\theoremstyle{remark}
\newtheorem{rem}{Remark}

% Marks the beginning of the document
\begin{document}
\bibliographystyle{IEEEtran}
\vspace{3cm}

\title{\textbf{Assignment-1}}
\author{EE24BTECH11052 - RONGALI CHARAN}
\maketitle
\bigskip

\renewcommand{\thefigure}{\theenumi}
\renewcommand{\thetable}{\theenumi}
\onecolumn
\setlength{\columnsep}{2.5em}
\section*{\textbf{SECTION-A}}
\section*{\textbf{JEE ADVANCED}}
\begin{enumerate}
\subsection*{E - Subjective Problems}
    \item If $f(x-y)=f(x).g(y)-f(y).g(x)$ and $ g(x-y)=g(x).g(y)-f(x).f(y) $ for all $x,y \in R $ . If right hand derivative at $x=0$ exists for $f(x)$. Find Derivative of $g(x)$ at $x=0$.
    \hfill(2005 - 4 Marks)
\subsection*{F.  Match the Following}
		 \item In this question there are entries in columns I and II. Each entry in \textbf{Column I} is related to exactly one entry in \textbf{Column II}. Write the correct letter from \textbf{Column II} against the entry number in \textbf{Column I}in your answer book.      \hfill(2009 - 4 Marks)
			\begin{multicols}{2}
                 \textbf{Column I}
				\begin{enumerate}[label=(\Alph*)]
				\item	$sin(\pi[x])$
				\item $sin(\pi(x-[x]))$
				\end{enumerate}
			\columnbreak
                 \textbf{Column II}
				\begin{enumerate}[label=(\alph*), start=16]
					\item  differentiable everywhere 
						
					\item  nowhere differentiable

					\item  not differentiable at 1 and -1  
						
				\end{enumerate}
				\end{multicols}
			\item In the following $[x]$ denotes the greatest integer less than or equal to x. Match the functions in Column I with the properties in column II and indicate your answer by darkening the appropriate bubbles in the $4\times4$ matrix given in ORS.
                  \hfill(2007 - 6 Marks)
			\begin{multicols}{2}
				\textbf{Column I}

				\begin{enumerate}[label=(\Alph*)]
					\item  $x|x|$ 
					\item  $\sqrt{|x|}$ 
					\item $x+[x]$ 
					\item  $|x-1|+|x+1|$
				\end{enumerate}
			\columnbreak
				\textbf{Column II}
				\begin{enumerate}[label=(\alph*), start=16]
					\item  continuous in $(-1,1)$

					\item differentiable in $(-1,1)$

					\item strictly increasing in $(-1,1)$

					\item  not differentiable atleast at one point in  $(-1,1)$
				\end{enumerate}
			\end{multicols}
		\item Let $f_1:R\rightarrow R$ $f_2:[0,\infty)\rightarrow R$ $f_3:R\rightarrow R$ $f_4:[0,\infty)\rightarrow R$ be defined by 
         $f_1(x) =
        \begin{cases}
                 |x| & \text{if } x < 0 \\
                 e^x & \text{if } x \geq 0 
                 \end{cases}$
                 ;$f_2(x)=x^2$ ; $g(x) =
                 \begin{cases}
                    \sin(x) & \text{if }  x < 0 \\
                    \ x & \text{if }  x\geq 0
                  \end{cases}$
                  ;$f_4(x) =
                  \begin{cases}
                     f_2(f_1(x)) & \text{if } x < 0 \\
                     f_2(f_1(x))-1 & \text{if } x \geq 0 
                  \end{cases}$
                  \hfill(JEE Adv. 2014)
                 \begin{multicols}{2} 
				\textbf{List-I} 
				\begin{enumerate}[label=\Alph*., start=16]
					\item $f_4$ is
					\item $f_3$ is 
					\item $f_2of_1$ is 
					\item $f_2$ is
				\end{enumerate}
				\columnbreak
				\textbf{List-II}
				\begin{enumerate}
					\item[1.]  Onto but not one-one 

					\item[2.]  Neither continuous nor one-one 

					\item[3.]  Differentiable but not one-one 

					\item[4.]  Continuous and one-one
				\end{enumerate}
		\end{multicols}
		\begin{multicols}{2}
			\textbf{   P Q R S}
			\begin{enumerate}[label=(\alph*)]
				\item $ 3 1 4 2$
			\end{enumerate}
			\begin{enumerate}[label=(\alph*), start=3]
				\item$ 3 1 2 4$
			\end{enumerate}
			\columnbreak
			\textbf{   P Q R S }
			\begin{enumerate}[label=(\alph*), start=2]
				\item $1 3 4 2$ 
			\end{enumerate}
			\begin{enumerate}[label=(\alph*)]
				\item $1 3 2 4$
			\end{enumerate}
		\end{multicols}
            \item Let $f_1: \textbf{R}\rightarrow \textbf{R},f_2:(-\frac{\pi}{2},\frac{\pi}{2})\rightarrow \textbf{R}, f_3:(-1,e^\frac{\pi}{2}-2)\rightarrow \textbf{R}$ and $f_4: \textbf{R}\rightarrow \textbf{R}$ be defined by \\
            1. $f_1(x)=\sin({\sqrt{1-e^{-x^2}}})$, \\
            2. $f_2(x) =
             \begin{cases}
                 \frac{|sin x|}{\tan^{-1}x} & \text{if } x \neq 0 \\
                 e^x & \text{if } x = 0 
                 \end{cases}$,where the inverse trigonometric function $tan^{-1}x $ assumes value in $(-\frac{\pi}{2},\frac{\pi}{2})$,\\
            3. $f_3(x)=[\sin({log_e(x+2)})]$,where, for $t\in \textbf{R}$, $[t]$ denotes the greatest integer less than or equal to t, \\
            4.$f_4(x) =
        \begin{cases}
                 x^2\sin{\frac{1}{x}} & \text{if } x \neq 0 \\
                 0 & \text{if } x = 0 
                 \end{cases}$ .
                 \begin{multicols}{2} 
				\textbf{List-I} 
				\begin{enumerate}[label=\Alph*., start=16]
					\item The function $f_1$ is
					\item The function $f_2$ is 
					\item The function $f_3$ is 
					\item The function $f_4$ is
				\end{enumerate}
				\columnbreak
				\textbf{List-II}
				\begin{enumerate}
					\item  NOT continuous at x = 0 

					\item  continuous at x = 0 and NOT differentiable at x = 0
					\item differentiable at x = 0 and its derivative is NOT continuous at x = 0
					\item differentiable at x = 0 and its derivative is continuous at x = 0
				\end{enumerate}
                 \hfill(JEE Adv. 2018)
		\end{multicols}
            \begin{multicols}{2}
			\begin{enumerate}[label=(\alph*)]
				\item $P\rightarrow2;Q\rightarrow3;R\rightarrow1;S\rightarrow4$
			\end{enumerate}
			\begin{enumerate}[label=(\alph*), start=3]
				\item $P\rightarrow4;Q\rightarrow2;R\rightarrow1;S\rightarrow3$
			\end{enumerate}
			\columnbreak
			\begin{enumerate}[label=(\alph*), start=2]
				\item $P\rightarrow4;Q\rightarrow1;R\rightarrow2;S\rightarrow3$
			\end{enumerate}
			\begin{enumerate}[label=(\alph*)]
				\item $P\rightarrow2;Q\rightarrow1;R\rightarrow4;S\rightarrow3$
			\end{enumerate}
		\end{multicols}
\twocolumn
\subsection*{ I - Integer Value Correct Type }
    \item Let $f:[1,\infty)\rightarrow [2,\infty)$ be a differentiable function such that $f(1)=2$. If 6$\int_1^x f(t)dt = 3xf(x)-x^3$ for all $x\geq1$. Then the value of $f(2)$ is 
    \hfill(2011)
   \item The largest value of non-negative integer a for which $\lim_{x \to 1}\{\frac{-ax+sin(x-1)+a}{x+sin(x-1)-1}\}^{\frac{1-x}{1-\sqrt{x}}} = \frac{1}{4}$ 
 \hfill(JEE Adv. 2014)
    \item Let $f:R\rightarrow R$ and $g:R\rightarrow R$ be respectively given by $f(x)=|x|+1$ and $g(x)=x^2+1$. Define $h:R\rightarrow R$ by \\
    $h(x) =
\begin{cases}
    $max\{f(x),g(x)\}$ & \text{if } x \leq 0 \\
    $min\{f(x),g(x)\}$ & \text{if } x > 0 
\end{cases}$.\\
The number of points at which $h(x)$ is not differentiable is 
\hfill(JEE Adv. 2014)
    \item Let m and n be two positive integers greater than 1. If $\lim_{\alpha\to 0}(\frac{e^{\cos{(\alpha^n)}}-e}{\alpha^m}) = - (\frac{e}{2})$ then the value of $\frac{m}{n}$ is 
    \hfill(JEE Adv. 2015)
    \item Let $ \alpha,\beta \in \textbf{R}$ be such that $\lim_{x \to 0}{\frac{x^2\sin{(\beta x)}}{\alpha x-sinx}} = 1$. Then $6(\alpha + \beta)$ equals. 
    \hfill(JEE Adv. 2016)
\section*{\textbf{SECTION-B}}
\section*{\textbf{JEE MAIN/AIEEE}}
    \item $\lim_{x\to 0}{\frac{\sqrt{1-\cos{2x}}}{\sqrt{2}x}}$ is
	    \hfill[2002]
    \begin{enumerate}[label=(\alph*)]
			\item $1$
			\item $-1$
			\item $0$
			\item does not exist
		\end{enumerate}
    \item  $\lim_{x\to \infty}({\frac{x^2+5x+3}{x^2+x+3}})^x$
	    \hfill[2002]
    \begin{enumerate}[label=(\alph*)]
			\item $e^4$
			\item $e^2$
			\item $e^3$
			\item $1$
		\end{enumerate}
    \item Let $f(x)=4$ and $f'(x)=4$. Then $\lim_{x\to 2}{\frac{xf(2)-2f(x)}{x-2}}$ is given by
	    \hfill[2002]
    \begin{enumerate}[label=(\alph*)]
			\item $2$
			\item $-2$
			\item $-4$
			\item $3$
		\end{enumerate}
    \item $\lim_{n\to \infty}{\frac{1^p+2^p+3^p+.....+n^p}{n^{p+1}}}$ is
	    \hfill[2002]
    \begin{enumerate}[label=(\alph*)]
			\item $\frac{1}{p+1}$
			\item $\frac{1}{1-p}$
			\item $\frac{1}{p}-\frac{1}{p-1}$
			\item $\frac{1}{p+2}$
		\end{enumerate}
    \item $\lim_{x\to 0}{\frac{\log x^n- [x]}{[x]}}, n\in N, $( $[x]$ denotes greatest integer less than or equal to x)
	    \hfill[2002]
    \begin{enumerate}[label=(\alph*)]
			\item has value $-1$
			\item has value $0$
			\item has value $1$
			\item does not exist
		\end{enumerate}

\end{enumerate}

\end{document}



